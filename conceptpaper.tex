\documentclass[10pt,a4paper]{article}
\begin{document}
\author{KABUGO IVAN, NDAGGA NICHOLAS,\\ SSENKAAYI CHARLES, MAZAPKWE LIDIAN}
\title{\textbf{BISIMILARITY IN LOGIC FOR STRATEGIC REASONING}}
\maketitle
\section{INTRODUCTION}

\section{BACKGROUND}	


\section{ LITERATURE REVIEW }
\section{GOALS AND OBJECTIVES}	
Our work makes three key contributions to the area of logics for multi-agent systems:
\subsection{First}
CATL is, to the best of our knowledge, the first logic 1Note that this is a rather different sense of the term commitment to that which is more commonly used in the multi-agent systems literature [16], in particular because commitment as represented in our counterfactual commitment operators is irrevocable.\\\\ We present our preliminary approach to a modal logic of revocable commitments
in another paper [15].which combines reasoning about strategic ability with counterfactual
reasoning.

\subsection{Second}
Although there has been previous work on logical characterisations of game-theoretic solution concepts, we believe that the combination of ability operators and the strategic counterfactual operator enables to express these properties much more elegantly and intuitively than has hitherto been possible.

\subsection{Third}
Our language extends ATL by introducing strategies as first-class components of the language, in much the same way that programs are first class components of the language of dynamic logic [12].\\\\ The resulting language not only enables one to reason about what coalitions can achieve, but also how they can achieve them. As we shall see in Section 4, the ability to name strategies explicitly within the language seems essential if we are to express properties such as Nash equilibrium.

\section{RESEARCH SCOPE}
In this paper, we did not include proofs of propositions and CATL properties due to lack of space.


\section{METHODOLOGY AND TIMELINES}
The remainder of the paper is structured as follows. 
We begin by introducing Action-based Alternating Transition Systems (AATSs)  which are used to give a semantics to CATL. Next, we describe the formal syntax and semantics of CATL and show that the model checking problem for CATL is tractable (i.e., can be solved in deterministic polynomial time). \\\\To illustrate the power of the logic, we introduce a simple formal model of games, and show how CATL can be used to reason about such games. In particular, we define a notion of correspondence between games and models in the logic, and show how game-theoretic concepts such as dominated strategies, Pareto optimality, and Nash equilibrium can be expressed as formulas of CATL.\\\\ Finally, we present some conclusions. We don’t include proofs of propositions and CATL properties due to lack of space.

\section{REFERENCES}


R. Alur, T. A. Henzinger, and O. Kupferman. Alternating-time Temporal Logic. In Proceedings of FOCS, pages 100–109, 1997.  R. Alur, T. A.\\\\Henzinger, and O. Kupferman.
Alternating-time temporal logic. Journal of the ACM, 49(5):672–713, September 2002.
 
	
\end{document}