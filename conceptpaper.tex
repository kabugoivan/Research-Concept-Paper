\documentclass[10pt,a4paper]{article}
\begin{document}
\author{KABUGO IVAN, NDAGGA NICHOLAS,\\ SSENKAAYI CHARLES, MAZAPKWE LIDIAN}
\title{\textbf{BISIMILARITY IN LOGIC FOR STRATEGIC REASONING}}
\maketitle
\section{INTRODUCTION}

\section{BACKGROUND}	


\section{ LITERATURE REVIEW }
Strategic reasoning in games concerns the plans or strategies that information processing
agents have for achieving certain goals. Strategy is one of the basic ingredients
of multi-agent interaction. It is basically the plan of action an agent (or a group
of agents) considers for its interactions, that can be modelled as games.\\\\ From the
game-theoretic viewpoint, a strategy of a player can be defined as a partial function
from the set of histories (sequences of events) at each stage of the game to the set of
actions of the player when it is supposed to make a move (Osborne and Rubinstein
1994).\\\\ Agents devise their strategies so as to force maximal gain in the game.
In cognitive science, the term ‘strategy’ is used much more broadly than in game
theory.Awell-known example is formed byGeorge Polya’s problem solving strategies
(understanding the problem, developing a plan for a solution, carrying out the plan, and
looking back to see what can be learned) (Polya 1945).



\section{GOALS AND OBJECTIVES}	
Our work makes three key contributions to the area of logics for multi-agent systems:
\subsection{First}

\subsection{Second}

\subsection{Third}

\section{RESEARCH SCOPE}
In this paper, we did not include proofs of propositions and CATL properties due to lack of space.


\section{METHODOLOGY AND TIMELINES}

\section{REFERENCES}


R. Alur, T. A. Henzinger, and O. Kupferman. Alternating-time Temporal Logic. In Proceedings of FOCS, pages 100–109, 1997.  R. Alur, T. A.\\\\Henzinger, and O. Kupferman.
Alternating-time temporal logic. Journal of the ACM, 49(5):672–713, September 2002.
 
	
\end{document}