\documentclass[10pt,a4paper]{article}
\begin{document}
\author{KABUGO IVAN, NDAGGA NICHOLAS,\\ SSENKAAYI CHARLES, MAZAPKWE LIDIAN}
\title{\textbf{BISIMILARITY IN LOGIC FOR STRATEGIC REASONING}}
\maketitle
\section{INTRODUCTION}

\section{BACKGROUND}	
Our aim in this paper is to present and evaluate a logic that supports precisely this kind of reasoning. CATL (which stands for Counterfactual ATL, but it can be also read as Commitment ATL) is based on ATL, the Alternating-time Temporal Logic of Alur, Henzinger, and Kupferman [1, 2], a logic which supports reasoning about the abilities of agents and coalitions of agents in game-like multi-agent systems.\\\\ CATL extends ATL with ternary counterfactual commitment operators, of the form , with the intended reading “if it were the case that agent i committed to strategy , then ' would hold”. \\\\The operators are counterfactual because they involve a supposition (that agent i commits to following strategy ) which is not known to be true or false; we say they are commitment operators because they capture the notion of an agent committing to follow a particular strategy.1 A formula will be true in a state q of a system  is true at state  in the system, where is exactly like M except that agent i is only able to perform the actions dictated by strategy .


\section{ LITERATURE REVIEW }
\section{GOALS AND OBJECTIVES}	


\section{METHODOLOGY AND TIMELINES}

\section{REFERENCES}


R. Alur, T. A. Henzinger, and O. Kupferman. Alternating-time Temporal Logic. In Proceedings of FOCS, pages 100–109, 1997.  R. Alur, T. A.\\\\Henzinger, and O. Kupferman.
Alternating-time temporal logic. Journal of the ACM, 49(5):672–713, September 2002.
 
	
\end{document}